%% bare_conf.tex
%% V1.4b
%% 2015/08/26
%% by Michael Shell
%% See:
%% http://www.michaelshell.org/
%% for current contact information.
%%
%% This is a skeleton file demonstrating the use of IEEEtran.cls
%% (requires IEEEtran.cls version 1.8b or later) with an IEEE
%% conference paper.
%%
%% Support sites:
%% http://www.michaelshell.org/tex/ieeetran/
%% http://www.ctan.org/pkg/ieeetran
%% and
%% http://www.ieee.org/

%%*************************************************************************
%% Legal Notice:
%% This code is offered as-is without any warranty either expressed or
%% implied; without even the implied warranty of MERCHANTABILITY or
%% FITNESS FOR A PARTICULAR PURPOSE! 
%% User assumes all risk.
%% In no event shall the IEEE or any contributor to this code be liable for
%% any damages or losses, including, but not limited to, incidental,
%% consequential, or any other damages, resulting from the use or misuse
%% of any information contained here.
%%
%% All comments are the opinions of their respective authors and are not
%% necessarily endorsed by the IEEE.
%%
%% This work is distributed under the LaTeX Project Public License (LPPL)
%% ( http://www.latex-project.org/ ) version 1.3, and may be freely used,
%% distributed and modified. A copy of the LPPL, version 1.3, is included
%% in the base LaTeX documentation of all distributions of LaTeX released
%% 2003/12/01 or later.
%% Retain all contribution notices and credits.
%% ** Modified files should be clearly indicated as such, including  **
%% ** renaming them and changing author support contact information. **
%%*************************************************************************


% *** Authors should verify (and, if needed, correct) their LaTeX system  ***
% *** with the testflow diagnostic prior to trusting their LaTeX platform ***
% *** with production work. The IEEE's font choices and paper sizes can   ***
% *** trigger bugs that do not appear when using other class files.       ***                          ***
% The testflow support page is at:
% http://www.michaelshell.org/tex/testflow/



\documentclass[conference]{IEEEtran}
% Some Computer Society conferences also require the compsoc mode option,
% but others use the standard conference format.
%
% If IEEEtran.cls has not been installed into the LaTeX system files,
% manually specify the path to it like:
%\documentclass[conference]{../sty/IEEEtran}





% Some very useful LaTeX packages include:
% (uncomment the ones you want to load)


% *** MISC UTILITY PACKAGES ***
%
%\usepackage{ifpdf}
% Heiko Oberdiek's ifpdf.sty is very useful if you need conditional
% compilation based on whether the output is pdf or dvi.
% usage:
% \ifpdf
%   % pdf code
% \else
%   % dvi code
% \fi
% The latest version of ifpdf.sty can be obtained from:
% http://www.ctan.org/pkg/ifpdf
% Also, note that IEEEtran.cls V1.7 and later provides a builtin
% \ifCLASSINFOpdf conditional that works the same way.
% When switching from latex to pdflatex and vice-versa, the compiler may
% have to be run twice to clear warning/error messages.






% *** CITATION PACKAGES ***
%
%\usepackage{cite}
% cite.sty was written by Donald Arseneau
% V1.6 and later of IEEEtran pre-defines the format of the cite.sty package
% \cite{} output to follow that of the IEEE. Loading the cite package will
% result in citation numbers being automatically sorted and properly
% "compressed/ranged". e.g., [1], [9], [2], [7], [5], [6] without using
% cite.sty will become [1], [2], [5]--[7], [9] using cite.sty. cite.sty's
% \cite will automatically add leading space, if needed. Use cite.sty's
% noadjust option (cite.sty V3.8 and later) if you want to turn this off
% such as if a citation ever needs to be enclosed in parenthesis.
% cite.sty is already installed on most LaTeX systems. Be sure and use
% version 5.0 (2009-03-20) and later if using hyperref.sty.
% The latest version can be obtained at:
% http://www.ctan.org/pkg/cite
% The documentation is contained in the cite.sty file itself.






% *** GRAPHICS RELATED PACKAGES ***
%
\ifCLASSINFOpdf
   \usepackage[pdftex]{graphicx}
  % declare the path(s) where your graphic files are
   \graphicspath{{img/}}
  % and their extensions so you won't have to specify these with
  % every instance of \includegraphics
   \DeclareGraphicsExtensions{.pdf,.jpeg,.png}
\else
  % or other class option (dvipsone, dvipdf, if not using dvips). graphicx
  % will default to the driver specified in the system graphics.cfg if no
  % driver is specified.
  % \usepackage[dvips]{graphicx}
  % declare the path(s) where your graphic files are
  % \graphicspath{{../eps/}}
  % and their extensions so you won't have to specify these with
  % every instance of \includegraphics
  % \DeclareGraphicsExtensions{.eps}
\fi
% graphicx was written by David Carlisle and Sebastian Rahtz. It is
% required if you want graphics, photos, etc. graphicx.sty is already
% installed on most LaTeX systems. The latest version and documentation
% can be obtained at: 
% http://www.ctan.org/pkg/graphicx
% Another good source of documentation is "Using Imported Graphics in
% LaTeX2e" by Keith Reckdahl which can be found at:
% http://www.ctan.org/pkg/epslatex
%
% latex, and pdflatex in dvi mode, support graphics in encapsulated
% postscript (.eps) format. pdflatex in pdf mode supports graphics
% in .pdf, .jpeg, .png and .mps (metapost) formats. Users should ensure
% that all non-photo figures use a vector format (.eps, .pdf, .mps) and
% not a bitmapped formats (.jpeg, .png). The IEEE frowns on bitmapped formats
% which can result in "jaggedy"/blurry rendering of lines and letters as
% well as large increases in file sizes.
%
% You can find documentation about the pdfTeX application at:
% http://www.tug.org/applications/pdftex





% *** MATH PACKAGES ***
%
%\usepackage{amsmath}
% A popular package from the American Mathematical Society that provides
% many useful and powerful commands for dealing with mathematics.
%
% Note that the amsmath package sets \interdisplaylinepenalty to 10000
% thus preventing page breaks from occurring within multiline equations. Use:
%\interdisplaylinepenalty=2500
% after loading amsmath to restore such page breaks as IEEEtran.cls normally
% does. amsmath.sty is already installed on most LaTeX systems. The latest
% version and documentation can be obtained at:
% http://www.ctan.org/pkg/amsmath





% *** SPECIALIZED LIST PACKAGES ***
%
%\usepackage{algorithmic}
% algorithmic.sty was written by Peter Williams and Rogerio Brito.
% This package provides an algorithmic environment fo describing algorithms.
% You can use the algorithmic environment in-text or within a figure
% environment to provide for a floating algorithm. Do NOT use the algorithm
% floating environment provided by algorithm.sty (by the same authors) or
% algorithm2e.sty (by Christophe Fiorio) as the IEEE does not use dedicated
% algorithm float types and packages that provide these will not provide
% correct IEEE style captions. The latest version and documentation of
% algorithmic.sty can be obtained at:
% http://www.ctan.org/pkg/algorithms
% Also of interest may be the (relatively newer and more customizable)
% algorithmicx.sty package by Szasz Janos:
% http://www.ctan.org/pkg/algorithmicx




% *** ALIGNMENT PACKAGES ***
%
%\usepackage{array}
% Frank Mittelbach's and David Carlisle's array.sty patches and improves
% the standard LaTeX2e array and tabular environments to provide better
% appearance and additional user controls. As the default LaTeX2e table
% generation code is lacking to the point of almost being broken with
% respect to the quality of the end results, all users are strongly
% advised to use an enhanced (at the very least that provided by array.sty)
% set of table tools. array.sty is already installed on most systems. The
% latest version and documentation can be obtained at:
% http://www.ctan.org/pkg/array


% IEEEtran contains the IEEEeqnarray family of commands that can be used to
% generate multiline equations as well as matrices, tables, etc., of high
% quality.




% *** SUBFIGURE PACKAGES ***
\ifCLASSOPTIONcompsoc
  \usepackage[caption=false,font=normalsize,labelfont=sf,textfont=sf]{subfig}
\else
  \usepackage[caption=false,font=footnotesize]{subfig}
\fi
% subfig.sty, written by Steven Douglas Cochran, is the modern replacement
% for subfigure.sty, the latter of which is no longer maintained and is
% incompatible with some LaTeX packages including fixltx2e. However,
% subfig.sty requires and automatically loads Axel Sommerfeldt's caption.sty
% which will override IEEEtran.cls' handling of captions and this will result
% in non-IEEE style figure/table captions. To prevent this problem, be sure
% and invoke subfig.sty's "caption=false" package option (available since
% subfig.sty version 1.3, 2005/06/28) as this is will preserve IEEEtran.cls
% handling of captions.
% Note that the Computer Society format requires a larger sans serif font
% than the serif footnote size font used in traditional IEEE formatting
% and thus the need to invoke different subfig.sty package options depending
% on whether compsoc mode has been enabled.
%
% The latest version and documentation of subfig.sty can be obtained at:
% http://www.ctan.org/pkg/subfig




% *** FLOAT PACKAGES ***
%
%\usepackage{fixltx2e}
% fixltx2e, the successor to the earlier fix2col.sty, was written by
% Frank Mittelbach and David Carlisle. This package corrects a few problems
% in the LaTeX2e kernel, the most notable of which is that in current
% LaTeX2e releases, the ordering of single and double column floats is not
% guaranteed to be preserved. Thus, an unpatched LaTeX2e can allow a
% single column figure to be placed prior to an earlier double column
% figure.
% Be aware that LaTeX2e kernels dated 2015 and later have fixltx2e.sty's
% corrections already built into the system in which case a warning will
% be issued if an attempt is made to load fixltx2e.sty as it is no longer
% needed.
% The latest version and documentation can be found at:
% http://www.ctan.org/pkg/fixltx2e


%\usepackage{stfloats}
% stfloats.sty was written by Sigitas Tolusis. This package gives LaTeX2e
% the ability to do double column floats at the bottom of the page as well
% as the top. (e.g., "\begin{figure*}[!b]" is not normally possible in
% LaTeX2e). It also provides a command:
%\fnbelowfloat
% to enable the placement of footnotes below bottom floats (the standard
% LaTeX2e kernel puts them above bottom floats). This is an invasive package
% which rewrites many portions of the LaTeX2e float routines. It may not work
% with other packages that modify the LaTeX2e float routines. The latest
% version and documentation can be obtained at:
% http://www.ctan.org/pkg/stfloats
% Do not use the stfloats baselinefloat ability as the IEEE does not allow
% \baselineskip to stretch. Authors submitting work to the IEEE should note
% that the IEEE rarely uses double column equations and that authors should try
% to avoid such use. Do not be tempted to use the cuted.sty or midfloat.sty
% packages (also by Sigitas Tolusis) as the IEEE does not format its papers in
% such ways.
% Do not attempt to use stfloats with fixltx2e as they are incompatible.
% Instead, use Morten Hogholm'a dblfloatfix which combines the features
% of both fixltx2e and stfloats:
%
% \usepackage{dblfloatfix}
% The latest version can be found at:
% http://www.ctan.org/pkg/dblfloatfix




% *** PDF, URL AND HYPERLINK PACKAGES ***
%
\usepackage{url}
% url.sty was written by Donald Arseneau. It provides better support for
% handling and breaking URLs. url.sty is already installed on most LaTeX
% systems. The latest version and documentation can be obtained at:
% http://www.ctan.org/pkg/url
% Basically, \url{my_url_here}.
\usepackage{hyperref}



% *** Do not adjust lengths that control margins, column widths, etc. ***
% *** Do not use packages that alter fonts (such as pslatex).         ***
% There should be no need to do such things with IEEEtran.cls V1.6 and later.
% (Unless specifically asked to do so by the journal or conference you plan
% to submit to, of course. )


% correct bad hyphenation here
\hyphenation{op-tical net-works semi-conduc-tor}


\begin{document}
%
% paper title
% Titles are generally capitalized except for words such as a, an, and, as,
% at, but, by, for, in, nor, of, on, or, the, to and up, which are usually
% not capitalized unless they are the first or last word of the title.
% Linebreaks \\ can be used within to get better formatting as desired.
% Do not put math or special symbols in the title.
\title{Overview on UWB\\standard 802.15.4-2024}


% author names and affiliations
% use a multiple column layout for up to three different
% affiliations
\author{\IEEEauthorblockN{Gabriele Da Re}
\IEEEauthorblockA{School of Computer Engineering\\
University of Padua\\
Padua, Veneto, Italy\\
Email: gabriele.dare00@gmail.com\\
Mat: 2158818
}}
%\and
%\IEEEauthorblockN{Homer Simpson}
%\IEEEauthorblockA{Twentieth Century Fox\\
%Springfield, USA\\
%Email: homer@thesimpsons.com}
%\and
%\IEEEauthorblockN{James Kirk\\ and Montgomery Scott}
%\IEEEauthorblockA{Starfleet Academy\\
%San Francisco, California 96678--2391\\
%Telephone: (800) 555--1212\\
%Fax: (888) 555--1212}}

% conference papers do not typically use \thanks and this command
% is locked out in conference mode. If really needed, such as for
% the acknowledgment of grants, issue a \IEEEoverridecommandlockouts
% after \documentclass

% for over three affiliations, or if they all won't fit within the width
% of the page, use this alternative format:
% 
%\author{\IEEEauthorblockN{Michael Shell\IEEEauthorrefmark{1},
%Homer Simpson\IEEEauthorrefmark{2},
%James Kirk\IEEEauthorrefmark{3}, 
%Montgomery Scott\IEEEauthorrefmark{3} and
%Eldon Tyrell\IEEEauthorrefmark{4}}
%\IEEEauthorblockA{\IEEEauthorrefmark{1}School of Electrical and Computer Engineering\\
%Georgia Institute of Technology,
%Atlanta, Georgia 30332--0250\\ Email: see http://www.michaelshell.org/contact.html}
%\IEEEauthorblockA{\IEEEauthorrefmark{2}Twentieth Century Fox, Springfield, USA\\
%Email: homer@thesimpsons.com}
%\IEEEauthorblockA{\IEEEauthorrefmark{3}Starfleet Academy, San Francisco, California 96678-2391\\
%Telephone: (800) 555--1212, Fax: (888) 555--1212}
%\IEEEauthorblockA{\IEEEauthorrefmark{4}Tyrell Inc., 123 Replicant Street, Los Angeles, California 90210--4321}}




% use for special paper notices
%\IEEEspecialpapernotice{(Invited Paper)}




% make the title area
\maketitle

% As a general rule, do not put math, special symbols or citations
% in the abstract
\begin{abstract}
The purpose of this survey is to understand and address these topics of the standard IEEE 802.15.4-2024,
"IEEE Standard for Low-Rate wireless networks":
Architecture, where we will s

\end{abstract}

% no keywords




% For peer review papers, you can put extra information on the cover
% page as needed:
% \ifCLASSOPTIONpeerreview
% \begin{center} \bfseries EDICS Category: 3-BBND \end{center}
% \fi
%
% For peerreview papers, this IEEEtran command inserts a page break and
% creates the second title. It will be ignored for other modes.
\IEEEpeerreviewmaketitle



\section{Introduction}
% no \IEEEPARstart
The standard covers many physical layers and one Medium Access Control layer (MAC) for 
low rate wireless personal area networks (LR-WPAN).
There are some special applications such as Smart Utility Network, 
Rail Communications and Control, Radio Frequency Identification (RFID),
Medical Body Area Networks. Among many others it covers the one we are interested in,
the Ultra Wide Band (UWB) technology.
UWB is a technology generally defined like others in the standard as a Wireless Personal Area Network 
(WPAN). So we will give an quick overview of the standard and then focus on the UWB technology.

%\hfill mds
% 
%\hfill August 26, 2015

\section{General Description}
\subsection{Network Topologies}
Topologies for LR-WPAN are two, star and peer-to-peer.
In the star topology, the coordinator is the central device and the other devices are the
end devices. The coordinator is the only device that can communicate with the end devices,
and is usually wall powered, whilst end devices are battery powered.
Suited for home automation, personal health care and games.\\
The peer-to-peer topology is a network of devices that can communicate with each other,
thus allowing for more complex networks, such as mesh networks, using multiple hops, 
implemented at higher level, thus not discussed in this standard.
Suited for sensor networks, enabling smart agriculture, industrial control 
and monitoring and asset and inventory traking.
\begin{figure}[!h]
    \centering
    \includegraphics[width=1\linewidth]{topologies}
    \caption{Star and peer-to-peer topologies}
    \label{fig:topologies}
\end{figure}

Each indipendent PAN selects a UID (PAN Identifier) thus allowing for multiple PANs to coexist,
moreover each device in a PAN can communicate within with a short address, permits to
communicate also with another device from another PAN.\\

\subsection{Architecture}
The architecture is composed of three layers:
\begin{itemize}
    \item Physical Layer (PHY)
    \item Medium Access Control (MAC)
    \item Higher layers
\end{itemize}

Only PHY and MAC are defined in the standard, the higher layers, such as network,
that involves its configuration, message routing and manipulation are left to the implementer.
as well as the application layer.\\

\begin{figure}[!h]
  \centering
  \includegraphics[width=.5\linewidth]{layers}
  \caption{IEEE 802.15.4-2024 architecture}
  \label{fig:layers}
\end{figure}

\subsubsection{Physical Layer (PHY)}
The PHY layer has its main focus on the activation and deactivation of the radio transceiver,
energy detection, link quality indication, channel freq. selection, clear channel assessment,
precision ranging (UWB) and data transmission and reception.
In the specific case of High Rate Pulse repetition frequency UWB, it also serves the purpose 
of precision ranging.

\subsubsection{Medium Access Control (MAC)}
The MAC overlay provides 2 services:
\begin{itemize}
    \item Data service
    \item Management service
\end{itemize}
The data service is responsible for the MAC protocol data units transmission and reception,
whilst the management service is responsible for the interfacing with the MAC sublayer 
management entity service access point (MLME-SAP fig.\ref{fig:layers}).
In particular the MAC overlay provides the possibility to manage beacons, channel access,
association and disassociation, acknowledged frame delivery, guaranteed time slots management
and frame validation. In addition can provide security features (TODO: UWB?).

\subsection{Functional overview}
\subsubsection{Scheduled access}
Access is managed by different implementations of the superframe structure.\\
\textbf{Beacon superframe}, defined and sent by the coiordinator, dependant on beacons.
Can have an active and inactive portion, during the latter the coordinator is able
to enter low-power mode (sleep), thus saving energy.
Beacon transmission is executed at the beginning of each superframe by the coordinator,
in order to synchronize and identify the devices of the PAN. It can be avoided by the
coordinator bypassing the beacon transmission.
The Superframe Duration $\le$ Beacon Interval, is divided in two parts:
\begin{itemize}
    \item Contention Access Period (CAP)
    \item Contention Free Period (CFP)
\end{itemize}

\begin{figure}[!h]
    \centering
    \includegraphics[width=1\linewidth]{superframe}
    \caption{Beacon superframe with Guaranteed Time Slots (GTS) in the CFP}
    \label{fig:superframe}
\end{figure}

\textbf{Deterministic and synchronous multichannel extension} (DSME) multi-superframe, as 
in beacon, starts with the PAN coordinator sending an Enhanced Beacon frame, containing
DSME PAN Descriptor Information Element (IE). The multi-superframe is divided in
cycles of repeated superframes \ref{fig:multi-superframe}, composed as usual, 
of enanched beacon frame, CAP and CFP.\\

\begin{figure}[!h]
    \centering
    \includegraphics[width=1\linewidth]{DSME}
    \caption{DSME multi-superframe}
    \label{fig:multi-superframe}
\end{figure}

\textbf{Time Slotted Channel Hopping} (TSCH) sees the substitution of the superframe with a 
slotframe, also containing guaranteed or CSMA-CA periods. The difference is the shared notion 
of time between partecipants, thus allowing for automatic repetition of the slotframe, without 
involving beacon transmission. It can also communicate the device's assigned timeslot(s) in
the slotframe by beacon, but tipically is handled at higher layers.
Since all devices are synchronized and share channel information, they can hop over channels
decreasing interference and multipath fading, doing so in slotted channels to avoid collisions,
thus avoiding retransmissions, usefull in industrial environments.\\

\textbf{TWVS multichannel cluster tree PAN} (TMCTP).
A cluster tree network is a mesh of clusters, each with a coordinator,
that can communicate with other clusters, forming a tree. The easiest is a single cluster,
with a coordinator and end devices, but can be extended to multiple clusters
by the first PAN coordinator, that instructs a device to become a coordinator 
of a new cluster. This augment the coverage area, with the downside of augmenting the
message latency.
The TMCTP is a cluster tree network with a Master PAN coordinator(Super Pan Coordinator), that
synchronizes other PAN coordinators over different channels, that in turn synchronize
their clusters.
Parent PAN coordinator(s) communicate with their PAN-coordinator child(s) in its own channel 
CAP or CFP, whilst childs send beacons to their parent in a dedicated channel,
Dedicated Beacon Slot (DBS) assigned by the coordinator in the Beacon Only Phase (BOP).
So the TMCTP has an enhanced superframe structure, with a BOP, a CAP and a CFP.\\

\subsubsection{Data Transfer Model}
The transfer models of the standard are:
\begin{itemize}
    \item Transfer to a Coordinator from a device.
    \item Indirect transfer from a coordinator in which the device recives the data.
    \item Transfer between two peer devices. (TODO: UWB?)
\end{itemize}

On a correctly recived frame, if requested, the receiver sends an acknowledgment frame that 
can be of 3 types:
\begin{itemize}
    \item Immediate acknowledgment
    \item Enanched acknowledged
    \item Fragment acknowledgment
\end{itemize}

\textbf{Data transfer to a coordinator} is managed in two ways depending on beacon enabled 
or not. If synchronization \emph{beacons are enabled}, the device listens for the beacon, when found 
it sinchronizes to the PAN, and sends the data frame to the coordinator at an appropriate 
time.\\
If \emph{not enabled} it transmits directly to the coordinator.\\

\textbf{Indirect data transfer} using the superframe structure, the coordinator that has data for
a device, indicates in the beacon that a data message is pending. Since devices are
synchronized through the beacon, they can listen for pending messages, if present, they
send a Data Request command to the coordinator, that in turn sends the data frame and when
succesfuly completed, removes the message from the pending list in the beacon.\\
If not using the superframe structure, and a Data Frame is pending, the coordinator stores 
the data and sends it to the device up on request by the latter. Else the coordinator that 
has no data, either indicates it on the returned ACK if requested by the device that sent the
Data Request, or in a Data Frame with zero payload.\\ 

Data transfer \textbf{between two peer} devices is managed either by a device that constantly 
recives or sichronizes with the sending device. In the first case the device attempts to 
send data when channel access is gained, in the second case other measures are taken to 
achieve sync.\\

\subsubsection{Frame Structure}
Thought to be reliable in noisy environments while keeping complexity low, the frames are
passed from the MAC to the PHY layer as the PHY Service Data Unit (PSDU), that is then
converted to the PHY Protocol Data Unit (PPDU) and transmitted. PPDU for HRP-UWB 
on UWB chapter.\\

\begin{figure}[!h]
  \centering
  \includegraphics[width=1\linewidth]{PPDU}
  \caption{PHY Protocol Data Unit (PPDU)}
  \label{fig:PPDU}
\end{figure}

\subsubsection{Information Elements}
To transmit information between layers and devices, the standard uses Information Elements 
(IEs), that consist of an ID and a length field, followed by the information itself. If the
IE is not recognized, it is ignored, else it can be accepted or discarded.\\

\subsubsection{Access Methods}
The standard defines the following access methods:
\begin{itemize}
    \item Unslotted carrier sense multiple access with collision avoidance (CSMA-CA), 
          used where is not used the superframe structure.
    \item Slotted CSMA-CA, used in the superframe structure.
    \item TSCH CCA (TSCH Clear Channel Assessment) in non shared slots where the MAC layer requests a CCA
          to the PHY layer at a designated time in the timeslot,
          that in turn returns the result.
    \item TSCH CSMA-CA, in shared slots (multiple devices can transmit in that assigned 
          timeslot 10.3.2.2-10.3.9.2 section of the standard). Collisions are detected by not
          reciving an acknowledgment frame. In shared links if there is a collision, 
          retransmission is implemented along with an exp. backoff mechanism to avoid further
          collisions (backoff exponent increased on every collision).
    \item CSMA-CA with Priority Channel Access (PCA), used in presence of MAC Service Data 
          Units containing CriticalEventMessage parameter flag set to true
    \item LECIM (Low Energy Critical Infrastructure Monitoring) ALOHA with PCA.
\end{itemize}

\textbf{Frame acknowledgment} is optionally sent if requested, to confirm reception and
validation of a frame. The reciving device can add content as enhanced acknowledgment frame
encapsulated as information elements, then if the originator does not understand the IE 
content of the Enh-Ack, is ignored but considering the transmission succesful.\\

\textbf{Frak} is used in a fragment sequence to determine which fragments have been recived
and which are missing explicitating the status of one or more fragments.\\

\begin{figure} [!h]
  \centering
  \includegraphics[width=.7\linewidth]{FRAK}
  \caption{Frak format}
  \label{fig:ack}
\end{figure}


\textbf{Data verification} is achieved in MAC service data units through a cyclic redundancy
check (CRC). For fragment sequences, is implemented a fragment identity check sequence (FICS)
included in each fragment, used also to determine along with the fragment number, which
fragments are missing and which are recived.\\

\begin{figure}[!h]
  \centering
  \includegraphics[width=.6\linewidth]{fragment-packet}
  \caption{Fragment packet}
  \label{fig:fragment}
\end{figure}

\subsubsection{Power Consumption}
Mainly, devices that are battery powered will require duty-cycling, that is
the device is active for a short period of time listening on the RF channel for incoming
messages and then goes to sleep the majority of the time, thus saving energy.
Sleep and listen periods are decided by the application designer, who finds a compromise
between message latency and battery consumption. Devices can also continously listen.\\

In URP UWB, the standard also provides a hybrid modulation that permits noncoherent 
architectures in order to reduce power consumption and implementation complexity.\\

\subsubsection{Security}
The cost objectives of ad nature of hoc networks impose additional security constraints,
but also result difficult to achieve. The problems are low cost devices, mostly with low
computational power, available memory and battery power. Also there is the problem of
the trusted computer base, or random number generation, that is difficult to achieve.
Is futhermore not implied that there is a fixed infrastructure, thus implying that are possible
communications between devices that have never communicated before.\\

Since most of the security features are implementable at higher layers, is out of scope
for the standard. (for further info IEEE 802.15.9).\\

The mechanism used in this standard is based on symmetric key cryptography, and uses keys
provided by the higher layer processes, that also provide establishment and maintenance of
the latter, thus the MAC layer is not involved in the key management and assumes that the
implementation is secure.\\

The mechanism provides combinations for 3 security services:
\begin{itemize}
    \item Data confidentiality: assumes informations are not disclosed to unauthorized parties.
    \item Data integrity: assures the data source, thus not altered during transmission.
    \item Replay protection: assures that a duplicate is detected. (important for UWB on 
          keyless devices).
\end{itemize}

The frame protection can be adapted on a per frame basis, varying the security level depending
on the requirement of security over security overhead. Keys are either shared between 2 
devices (link key) or between a group of devices (group key). This implies lower maintenance
and storage costs, but also implies that if adevice in the group is compromised, the whole
group is compromised.\\

\section{General PHY requirements}
\subsection{}

% An example of a floating figure using the graphicx package.
% Note that \label must occur AFTER (or within) \caption.
% For figures, \caption should occur after the \includegraphics.
% Note that IEEEtran v1.7 and later has special internal code that
% is designed to preserve the operation of \label within \caption
% even when the captionsoff option is in effect. However, because
% of issues like this, it may be the safest practice to put all your
% \label just after \caption rather than within \caption{}.
%
% Reminder: the "draftcls" or "draftclsnofoot", not "draft", class
% option should be used if it is desired that the figures are to be
% displayed while in draft mode.
%
%\begin{figure}[!t]
%\centering
%\includegraphics[width=2.5in]{myfigure}
% where an .eps filename suffix will be assumed under latex, 
% and a .pdf suffix will be assumed for pdflatex; or what has been declared
% via \DeclareGraphicsExtensions.
%\caption{Simulation results for the network.}
%\label{fig_sim}
%\end{figure}

% Note that the IEEE typically puts floats only at the top, even when this
% results in a large percentage of a column being occupied by floats.


% An example of a double column floating figure using two subfigures.
% (The subfig.sty package must be loaded for this to work.)
% The subfigure \label commands are set within each subfloat command,
% and the \label for the overall figure must come after \caption.
% \hfil is used as a separator to get equal spacing.
% Watch out that the combined width of all the subfigures on a 
% line do not exceed the text width or a line break will occur.
%
%\begin{figure*}[!t]
%\centering
%\subfloat[Case I]{\includegraphics[width=2.5in]{box}%
%\label{fig_first_case}}
%\hfil
%\subfloat[Case II]{\includegraphics[width=2.5in]{box}%
%\label{fig_second_case}}
%\caption{Simulation results for the network.}
%\label{fig_sim}
%\end{figure*}
%
% Note that often IEEE papers with subfigures do not employ subfigure
% captions (using the optional argument to \subfloat[]), but instead will
% reference/describe all of them (a), (b), etc., within the main caption.
% Be aware that for subfig.sty to generate the (a), (b), etc., subfigure
% labels, the optional argument to \subfloat must be present. If a
% subcaption is not desired, just leave its contents blank,
% e.g., \subfloat[].


% An example of a floating table. Note that, for IEEE style tables, the
% \caption command should come BEFORE the table and, given that table
% captions serve much like titles, are usually capitalized except for words
% such as a, an, and, as, at, but, by, for, in, nor, of, on, or, the, to
% and up, which are usually not capitalized unless they are the first or
% last word of the caption. Table text will default to \footnotesize as
% the IEEE normally uses this smaller font for tables.
% The \label must come after \caption as always.
%
%\begin{table}[!t]
%% increase table row spacing, adjust to taste
%\renewcommand{\arraystretch}{1.3}
% if using array.sty, it might be a good idea to tweak the value of
% \extrarowheight as needed to properly center the text within the cells
%\caption{An Example of a Table}
%\label{table_example}
%\centering
%% Some packages, such as MDW tools, offer better commands for making tables
%% than the plain LaTeX2e tabular which is used here.
%\begin{tabular}{|c||c|}
%\hline
%One & Two\\
%\hline
%Three & Four\\
%\hline
%\end{tabular}
%\end{table}


% Note that the IEEE does not put floats in the very first column
% - or typically anywhere on the first page for that matter. Also,
% in-text middle ("here") positioning is typically not used, but it
% is allowed and encouraged for Computer Society conferences (but
% not Computer Society journals). Most IEEE journals/conferences use
% top floats exclusively. 
% Note that, LaTeX2e, unlike IEEE journals/conferences, places
% footnotes above bottom floats. This can be corrected via the
% \fnbelowfloat command of the stfloats package.




\section{Conclusion}
The conclusion goes here.




% conference papers do not normally have an appendix


% use section* for acknowledgment
\section*{Acknowledgment}


The authors would like to thank...





% trigger a \newpage just before the given reference
% number - used to balance the columns on the last page
% adjust value as needed - may need to be readjusted if
% the document is modified later
%\IEEEtriggeratref{8}
% The "triggered" command can be changed if desired:
%\IEEEtriggercmd{\enlargethispage{-5in}}

% references section

% can use a bibliography generated by BibTeX as a .bbl file
% BibTeX documentation can be easily obtained at:
% http://mirror.ctan.org/biblio/bibtex/contrib/doc/
% The IEEEtran BibTeX style support page is at:
% http://www.michaelshell.org/tex/ieeetran/bibtex/
%\bibliographystyle{IEEEtran}
% argument is your BibTeX string definitions and bibliography database(s)
%\bibliography{IEEEabrv,../bib/paper}
%
% <OR> manually copy in the resultant .bbl file
% set second argument of \begin to the number of references
% (used to reserve space for the reference number labels box)
\begin{thebibliography}{1}

\bibitem{IEEEhowto:kopka}
H.~Kopka and P.~W. Daly, \emph{A Guide to \LaTeX}, 3rd~ed.\hskip 1em plus
  0.5em minus 0.4em\relax Harlow, England: Addison-Wesley, 1999.

\end{thebibliography}




% that's all folks
\end{document}


